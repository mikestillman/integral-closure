\documentclass[11pt]{amsart}
\usepackage{geometry}                % See geometry.pdf to learn the layout options. There are lots.
\geometry{letterpaper}                   % ... or a4paper or a5paper or ... 
%\geometry{landscape}                % Activate for for rotated page geometry
%\usepackage[parfill]{parskip}    % Activate to begin paragraphs with an empty line rather than an indent
\usepackage{graphicx}
\usepackage{amssymb}
\usepackage{epstopdf}
\DeclareGraphicsRule{.tif}{png}{.png}{`convert #1 `dirname #1`/`basename #1 .tif`.png}
\def\g{{\gamma}}
\def\mm{{\frak m}}
\title{Shamash Complex induction}
\author{DE+MES}
%\date{}                                           % Activate to display a given date or no date

\begin{document}
\maketitle
Tired of redoing the computation, I decided to TeX it:

\section{Conventions:} $B\subset Z$ and  $H$ for boundaries, cycles, homology of the Koszul complex $K$ of a  ring $R,\mm$ (local). $T$ for a basis element of the homology; $E$ for the free $R$-module $R^T$; 
$(a,b...)$ represents $K_a\otimes E_b\otimes \cdots$. Elements such as $k\otimes x_1\cdots$ are written without the $\otimes$.
Here $0\leq a\leq numgens\ R$; and $2\leq b,c\dots\leq 1+pd(R)$, where $pd(R)$ = the highest degree of the Koszul homology. The homological degree of $(a,b\dots$ is the sum; the weight is the
number of factors that are $>0$.

Note that $KK \subset K$; and $ZZ\subset H+B = Z$. Also $ZB\subset B$.

\section{Formulas, start of induction}
$d,\g$ are to be defined inductively on the weight (and degree??) using formulas:

\begin{enumerate}
\item $d:K\to K$ is the Koszul differential; $d: E \to K$ is the projection to $E/\mm E \cong H$.
\item 
$$
d(ku_1\cdots) = (dk)u_1\cdots +(-1)^kd(1u_1\cdots).
$$
\item
on $(0,b,c\dots)$, 
$$
d(x_1x_2\cdots x_n) = d(x_1\cdots x_{n-1})x_n+\g(x_1,\dots, x_n)
$$
with weight $\g(x_1,\dots, x_n)<n$.
\end{enumerate}

\section{weight 2} 
For $kx\in KE$ 
$$
d(kx) = (dk)x+(-1)^k k(dx) \in BE+K^2.
$$
Note that the weight does not decrease unless $k\in K_1$ or $k \in Z$.

For $x_1x_2\in E^2$
$$
d(x_1x_2)= (dx_1)x_2 +\g(x_1,x_2) = T_1x_2+\gamma(x_1,x_2)
$$
so 
$$
d\gamma(x_1,x_2) = -d((dx_1)x_2) = -T_1T_2 = T' = T+B  \in K;
$$
This lifts to $\g(x_1,x_2)=-x_{1,2}-z \in E+K$, 
and we see that $d:E^2\to HE+ E+K$.

\section{weight 3}
For $kx_1x_2\in KE^2$,
$$
d(kx_1x_2) = -[(dk)x_1x_2 + (-1)^k kd(x_1x_2)]
$$
so $d: KE^2 \to BE^2+KHE+KE+K^2 = BE^2+KE+K.$

For $x = x_1x_2x_3\in E^3$,
$$
dx = d(x_1x_2)x_3+\g(x) = T_1x_2x_3+\g(x_1x_2)x_3+\g(x), \hbox{ where $T_1 = dx_1$ }
$$
so
$$
d\g(x) = -[d(T_1x_2x_3)  + d(\g(x_1x_2)x_3)].
$$
and $d(T_1x_2x_3) = (-1)^{T_1}T_1d(x_2x_3)$.

Now $d(\g(x_1x_2)x_3)$ is a boundary by definition, and $\g(x_{1}x_{2})x_{3}$ has weight $2<3$. So
 $\g(x_1x_2x_3) = - (-1)^{T_{1}}G-\g(x_1x_2)x_3$, for some $G$  of weight $<3$ such that
$$
dG = d(T_1x_2x_3).
$$
By Lemma 3 there is a $w$ of weight 4 such that  $G := T_{1}x_{2}x_{3} -dw$ has weight $<3$; and
since $d^{2}w = 0$, this has the right property.

The problem is that we haven't defined $d$ on elements of weight 4 yet....


We have 
$$
d(T_1x_2x_3) = T_1d(x_2x_3) = T_1T_2x_3+T_1\g(x_{2}x_3) = T_1T_2x_3-T_1(x_{2,3}+k)
$$
Note that $T_1T_2 = d(x_{1,2}+k_{1,2})$.
Is the RHS actually a cycle??
$$
d(T_1T_2x_3+T_1(x_{2,3}+k)) = (-1)^{T_1T_2}T_1T_2T_3 + (-1)^{T_1}T_1T_2T_3 
$$
 
\end{document}  