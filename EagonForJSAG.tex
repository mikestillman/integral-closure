   
   
   \documentclass[11pt, oneside]{article}   	% use "amsart" instead of "article" for AMSLaTeX format
   \usepackage{geometry}                		% See geometry.pdf to learn the layout options. There are lots.
   \geometry{letterpaper}                   		% ... or a4paper or a5paper or ... 
   %\geometry{landscape}                		% Activate for rotated page geometry
   %\usepackage[parfill]{parskip}    		% Activate to begin paragraphs with an empty line rather than an indent
   \usepackage{graphicx}				% Use pdf, png, jpg, or eps§ with pdflatex; use eps in DVI mode
   								% TeX will automatically convert eps --> pdf in pdflatex		
   \usepackage{amssymb}
   
   %SetFonts
   
   %SetFonts
   
   
   \title{The EagonResolution package in Macaulay2}
   \author{David Eisenbud and Michael Stillman}
   %\date{}							% Activate to display a given date or no date
   
   \begin{document}
   \maketitle
   %\section{}
   %\subsection{}
   
   
   
  This package implements Eagon's algorithm for producing a not-necessarily minimal resolution of
   the residue field of a ring $R = S/I$ where $S$ is a polynomial ring and $I$ is an ideal.
   The resolution constructed is minimal if and only if R is Golod. The resolution
   constructed is sometimes called the Golod or Shamash or Eagon resolution.
   
   This resolution was described, in the special case where it is minimal, by
   E.S. Golod: Homology of some local rings, Uspekhi Mat. Nauk 33 (1978), no. 5(203), 177?178.
   A general construction was described by Jack Shamash:
   The Poincar� series of a local ring II, J. Algebra 17 (1971), 1?18
   and, perhaps around the same time, by Jack Eagon.
   Eagon's construction, superficially different than Shamash'
   was not published by him, but is described in Ch. 4 of the notes
   by Gulliksen and Levin: Homology of local rings,
   Queen's Paper in Pure and Applied Mathematics, No. 20 Queen's University, Kingston, Ont. 1969.  
   
   To get a glimpse of the construction, consider the first steps. Let 
   $K$ be the Koszul complex of $S$, which is the minimal $S$-free resolution
   of the residue field k. If numgens $S = n$, this begins 
   
  $$
   (K_1 = S^n) \to (K_0 = S) \to k.
   $$
 
   
   Let F be the mimimal $S$-free resolution of R.
   by the right-exactness of the tensor product, the complex
   
   $$
   R\otimes K_1 \to R\otimes K_0 \to k
   $$
   
   is a presentation of $k$, and of course $R\otimes K_2$ maps to the kernel of
   $R\otimes K_1 \to R \otimes K_0$. But there are new elements of the kernel, obtained by
   writing the generators of $I$, which correspond to the generators of $F_1$,
   in terms of the generators of the maximal ideal. Thus we must add a map
   $R\otimes F_1 \to R\otimes K_1$, and it is easy to show that the resulting complex
   
   $$
   R\otimes F_1 ++ R\otimes K_2 \to R\otimes K_1 \to R\otimes K_0 \to k
   $$
   
   is exact. There are three important points to note:
   
   1) $F_0$ does not occur
   
   2) $F_1$ occurs in homological degree 2

   3) There is a map $F_1 \to K_1$ that must be introduced and that does not
      come from either the complex $F$ nor the complex $K$.
      
   Eagon showed how this complex can be continued to a resolution.
   The underlying graded
   module of the complex is $K \otimes T(F')$, where $F'$ is the complex $F$, shifted by
   1 in homological degree so that $F_i$ is in homological degree $i+1$, and truncated
   by dropping $F_0$; and $T(F')$ denotes the tensor algebra on the graded module $F'$.

   The differentials of the complex come from the differentials in the Koszul
   complex and various maps identifying the homology, at successive stages of the 
   construction, with tensor products of modules already constructed.
   These are also the ingredients of
   the "Massey products" from topology, used by Golod to construct the complex
   in the special case where there are ``trivial Massey products'', and the resolution is therefore minimal.
   
   The command   eagon produces a type of hashTable called an  EagonData, defined in the package.
   It contains all the data produced in
   Eagon's construction of the resolution: a double complex $Y^n_i$, and some internal 
   maps. The vertical differental is called $dVert: Y^n_i \to Y^n_{i+1}$ and the horizontal
   differential is $dHor: Y^n_i \to Y^{n-1}_i. $

   Thus for example if $R$ is a factor ring of a polynomial ring $S$, then
   
\begin{verbatim}
 
   E = eagon(R,5)
    eagonResolution E

   or simply 

   res EE
\end{verbatim}

   produces the first 5 steps of a 
   (not necessarily minimal) $R$-free resolution of the residue field of $R$.  The function picture gives
   alternate ways of viewing the innards of the resolution.
   
   
\begin{verbatim}
 
   S = ZZ/101[a,b,c]
   I = ideal(a,b,c)*ideal(b,c)
   R = S/I
   E = eagon(R,5)
   F = eagonResolution E
   assert(F == res E)
 \end{verbatim}

   As stated above, $F = K\otimes T(F')$, and one can see the maps between 
   each pair of summands. We label the summand 
   $K_i\otimes F_{j_1}\otimes\cdots \otimes F_{j_m}$ with the symbol $(i,\{j_1,..,j_m\})$, and we can write out
   the differentials in block form with the function picture, 
   with the option 
   
\begin{verbatim}
 Display => "DisplayBlocks"
\end{verbatim}
 including the labels:
 
\begin{verbatim}
   F.dd_3
   picture(F.dd_3, Display => "DisplayBlocks")

\end{verbatim}
   Since the matrices can be very large, it is sometimes better to know just whether
   a given block is zero or not, and this can be obtained with the function  picture,
   with the default option 
   
\begin{verbatim}
    Display => "picture".
  Example   
   picture F.dd_3
   picture (F, Verbose => true)
   picture (F, Verbose => true, Transpose => true)
   \end{verbatim}
    \end{document}  